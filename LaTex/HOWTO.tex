%% Default Latex document template
%%
%%  blake@rcs.ee.washington.edu

\documentclass[letterpaper]{article}
\usepackage{hyperref}
% Uncomment for bibliog.
%\bibliographystyle{unsrt}

\usepackage{graphicx}
\usepackage{lineno}
%\usepackage{fancyhdr}

%%%%%%%%%%%%%%%%%%%%%%%%%%%%%%%%%%%%%%%%5
%
%  Set Up Margins 

%
%        Font selection
%
%\renewcommand{\rmdefault}{ptm}             % Times
%\renewcommand{\rmdefault}{phv}             % Helvetica
%\renewcommand{\rmdefault}{pcr}             % Courier
%\renewcommand{\rmdefault}{pbk}             % Bookman
%\renewcommand{\rmdefault}{pag}             % Avant Garde
%\renewcommand{\rmdefault}{ppl}             % Palatino
%\renewcommand{\rmdefault}{pch}             % Charter


%%%%%%%%%%%%%%%%%%%%%%%%%%%%%%%%%%%%%%%%%%%%%%%%%
%
%         Page format Mods HERE
%
%Mod's to page size for this document
\addtolength\textwidth{0cm}
\addtolength\oddsidemargin{0cm}
\addtolength\headsep{0cm}
\addtolength\textheight{0cm}
%\linespread{0.894}   % 0.894 = 6 lines per inch, 1 = "single",  1.6 = "double"
 

% Make table rows deeper
%\renewcommand\arraystretch{2.0}% Vertical Row size, 1.0 is for standard spacing)

\begin{document}
\begin{center}
\section*{LaTex Output from IK-BT Package}
Blake Hannaford, Dianmu Zhang\\
University of Washington\\
July 2017
\end{center}

\section{Introduction}
The IK-BT package generates a report on your inverse kinematics solution automatically.   This produces an output that can be much easier to read and check.   The Latex output consists of two parts:
\begin{enumerate}
    \item {\tt ik\_report\_template.tex} --- the template document containing overall setup and formatting information,
    \item {\tt IK\_solution\_NAME.tex}  --- the specific solution equations produced for your robot (where {\tt NAME} is your robot's name), 
    \item {\tt IK\_solution.tex} --- a copy of the most recently solved {\tt IK\_solution\_NAME.tex}.
\end{enumerate}

\section{How to use}
To produce your readable output, just enter
\begin{verbatim}

> pdflatex ik_report_template.tex

\end{verbatim}
The template will automatically include your most recent solution ({\tt $\backslash$input\{IK\_solution.tex\}}).  {\tt IK\_solution\_NAME.tex} is kept so that solving a new robot does not clobber your old work. 

Your output will then appear as {\tt ik\_report\_template.pdf} .   
The package produces the following output sections in the Latex file:


\begin{enumerate}
    \item Introduction
    \item Kinematic Parameters:
    Lists your DH parameters for reference and error checking.
    \item Forward Kinematic Equations:
    These are the automatically computed FK equations that are actually solved. 
    \item Unknown Variables:
    Lists the unknows in solution order. 
    \item Solutions:
    The equations for all solutions for each variable. 
    \item Solution Branching graph:
    A graph illustrating the dependencies between solutions (but so far this is 
    only in text form).
    \item Solution Sets:  The valid solutions (e.g. manipulator poses) for a given end effector configuration (${T_6^0}$)
    are listed here.  
    
\end{enumerate}


Sometimes the equations or the graph can be too long and extend beyond the right side margin.   In this case you need to go into the {\tt IK\_solution.tex} file and make appropriate edits to break up the equations onto separate lines.  Suggestions are provided in the report. 


\section{LaTex}
If you are not familiar with LaTex, you can install it in your system, or mouse the latex content into an online tool such as \url{https://www.overleaf.com}.

Debian Based Linux:   $>$ apt-get -y install texlive
  
Other OS:  \url{https://www.latex-project.org/get/}


\end{document}

